%%%%%%%%%%%%%%%%%%%%%%%%%%%%%%%%%%%%%%%%%%%%%%%%%%%%%%%%%%%%%%%%%%%%%%%%%%%%%%%%
%2345678901234567890123456789012345678901234567890123456789012345678901234567890
%        1         2         3         4         5         6         7         8

\documentclass[letterpaper, 10 pt, conference]{ieeeconf}  % Comment this line out if you need a4paper

%\documentclass[a4paper, 10pt, conference]{ieeeconf}      % Use this line for a4 paper

\IEEEoverridecommandlockouts                              % This command is only needed if 
                                                          % you want to use the \thanks command

\overrideIEEEmargins                                      % Needed to meet printer requirements.

% See the \addtolength command later in the file to balance the column lengths
% on the last page of the document
% The following packages can be found on http:\\www.ctan.org
\usepackage{mysymb}
%\usepackage{graphics} % for pdf, bitmapped graphics files
%\usepackage{epsfig} % for postscript graphics files
%\usepackage{mathptmx} % assumes new font selection scheme installed
%\usepackage{times} % assumes new font selection scheme installed
%\usepackage{amsmath} % assumes amsmath package installed
%\usepackage{amssymb}  % assumes amsmath package installed
\newcommand{\reffig}[1]{Fig. \ref{#1}}

\title{\LARGE \bf
Estimation and Control Library}


\author{Pan %<-this % stops a space
\thanks{$^{1}$ The Pennsylvania State University, University Park, PA 16802, USA.
        {\tt\small pan.liu@psu.edu}}%
}

\begin{document}

\maketitle
\thispagestyle{empty}
\pagestyle{empty}


%%%%%%%%%%%%%%%%%%%%%%%%%%%%%%%%%%%%%%%%%%%%%%%%%%%%%%%%%%%%%%%%%%%%%%%%%%%%%%%%
\begin{abstract}
A library for optimal estimation and control, as well as classical/modern control theories. Algorithms are applied on simulated robots using Ignition Gazebo.
\end{abstract}

%%%%%%%%%%%%%%%%%%%%%%%%%%%%%%%%%%%%%%%%%%%%%%%%%%%%%%%%%%%%%%%%%%%%%%%%%%%%%%%%

\section{INTRODUCTION} \label{introsec}
A library for optimal estimation and control, as well as classical/modern control theories. Algorithms are applied on simulated robots using ROS \cite{quigley2009ros} and Ignition Gazebo \cite{igngazebo}.



\section{CARTPOLE DYNAMICS} \label{carpoledyn}
This section derives the equation of motion of the cartpole system in the following two ways: (1) standard vector analysis; (2) using Lagrangian.

\subsection{Standard vector analysis}

The position of the cart is given as
\begin{equation}
    \mathbf{r}_1=x\mathbf{E}_x
\end{equation}
Computing the 1st and 2nd order rate of change of $\mathbf{r}_1$ in reference frame $\mathcal{F}_i$, we obtain the velocity and acceleration as
\begin{align}
    \mathbf{v}_1 &= ^{\mathcal{F}_i}\frac{d}{dt}(\mathbf{r}_1)=\dot{x}\mathbf{E}_x \\
    \mathbf{a}_1 &= ^{\mathcal{F}_i}\frac{d}{dt}(\mathbf{v}_1)=\ddot{x}\mathbf{E}_x
\end{align}


The position of the pole is given as
\begin{equation}
    \mathbf{r}_2=x\mathbf{E}_x + l\mathbf{e}_r
\end{equation}
Similarly, the velocity and acceleration of the pole can be computed as
\begin{align}
    \mathbf{v}_2 &= ^{\mathcal{F}_i}\frac{d}{dt}(\mathbf{r}_2) = \dot{x}\mathbf{E}_x + l\dot{\theta}\mathbf{e}_\theta\times\mathbf{e}_r=\dot{x}\mathbf{E}_x-l\dot{\theta}\mathbf{e}_z\\
    \mathbf{a}_2 &= ^{\mathcal{F}_i}\frac{d}{dt}(\mathbf{v}_2) = \ddot{x}\mathbf{E}_x -l\ddot{\theta}\mathbf{e}_z-l\dot{\theta}^2\mathbf{e}_r
\end{align}
Rewriten $\mathbf{a}_2$ in $\mathbf{E}_x$ and $\mathbf{E}_z$ as
\begin{equation}
    \mathbf{a}_2 = (\ddot{x} + l\ddot{\theta}\cos\theta - l\dot{\theta}^2\sin\theta)\mathbf{E}_x - (l\ddot{\theta}\sin\theta+l\dot{\theta}^2\cos\theta)\mathbf{E}_z
\end{equation}

Applying Netwon 2nd law on the $\mathbf{E}_x$-direciton of the cart and pole, we have
\begin{align}
    F - T_x &= m_1\ddot{x}\\
    T_x &= m_2(\ddot{x} + l\ddot{\theta}\cos\theta - l\dot{\theta}^2\sin\theta)
\end{align}
Adding the above two euqations we get the first equation,
\begin{equation}
    F = (m_1+m_2)\ddot{x} + m_2(l\ddot{\theta}\cos\theta - l\dot{\theta}^2\sin\theta)
\end{equation}

The cartpole system is consisted of two systems, i.e. the cart and the pole. The internal force can be canceled out by analyzing the two systems as a whole. A simplier way is to apply the balance of angular momemtun relative to the cart.
\begin{equation}
    \frac{d}{dt}\mathbf{H}_c = \mathbf{M}_c - (\mathbf{r}_2-\mathbf{r}_1)\times m_2\mathbf{a}_1
\end{equation}
where $\mathbf{H}_c$ is the angular momemtun of the cartpole system relative to the cart.
\begin{equation}
    \mathbf{H}_c = (\mathbf{r}_2-\mathbf{r}_1) \times m_2(\mathbf{v}_2 - \mathbf{v}_1) = m_2l^2\dot{\theta}\mathbf{E}_y
\end{equation}
And $\mathbf{M}_c$ is the moment relative to the cart.
\begin{align}
    \mathbf{M}_c = (\mathbf{r}_2-\mathbf{r}_1) \times (-m_2g)\mathbf{E}_z = m_2gl\sin\theta\mathbf{E}_y\\
    (\mathbf{r}_2-\mathbf{r}_1)\times m_2\mathbf{a}_1 = l\mathbf{e}_r \times m_2\ddot{x}\mathbf{E}_x = m_2l\ddot{x}\cos\theta\mathbf{E}_y
\end{align}
Thus, we get the second equation
\begin{equation}
    \cos\theta\ddot{x} + l\ddot{\theta} = g\sin\theta
\end{equation}

In summary, the equation of motion is
\begin{align}
    (m_1+m_2)\ddot{x} + m_2(l\ddot{\theta}\cos\theta - l\dot{\theta}^2\sin\theta) = F\\
    \cos\theta\ddot{x} + l\ddot{\theta} = g\sin\theta
\end{align}


\subsection{Linearization}
Assuming $\mathbf{q}=[x,\theta]^T$, the equations of motion of the cartpole system can be written in the standard form as
\begin{equation}
    \mathbf{H}(\mathbf{q})\ddot{\mathbf{q}} + \mathbf{C}(\mathbf{q},\mathbf{\dot{q}})\mathbf{\dot{q}} + \mathbf{G}(\mathbf{q}) = \mathbf{B}\mathbf{u}
\end{equation}
where $\mathbf{q}=[x,\theta]^T$ is a $n$-vector called the \textit{generalized coordinates vector}, $\mathbf{H}(\mathbf{q})$ is a $n \times n$ nonsingular symmetric positive-definite matrix called the \textit{mass matrix}, $\mathbf{C}(\mathbf{q},\mathbf{\dot{q}})$ is a $n \times n$ matrix called the \textit{centrifugal/Coriolis/friction matrix}, $\mathbf{G}(\mathbf{q})$ is a $n$-vector sometimes called the \textit{conservative forces vector}. \textbf{This equation of motion is valid for systems that follow classical Newton-Euler mechanics or Lagrangian mechanics} with a kinetic energy that is quadratic in the derivative of the generalized coordinates and a potential energy that may depend on the generalized coordinates, but not on its derivative. Such systems include robot arms, mobile robots, airplanes, helicopters,underwatervehicles,hovercraft,etc. 
\begin{equation}
\begin{split}
    \mathbf{H}(\mathbf{q}) &= \left[ \begin{array}{cc}
        m_1+m_2 & m_2l\cos\theta \\ 
        \cos\theta & l \\ 
        \end{array} \right], 
    \mathbf{C}(\mathbf{q},\mathbf{\dot{q}}) = \left[ \begin{array}{cc} 
        0 & -m_2l\dot{\theta}\sin\theta \\ 
        0 & 0 \\ 
        \end{array} \right], \\
    \mathbf{G}(\mathbf{q}) &= \left[ \begin{array}{c}
        0 \\ 
        -g\sin\theta \\ 
        \end{array} \right],
    \mathbf{B} = \left[ \begin{array}{c}
        1 \\ 
        0 \\ 
        \end{array} \right],
\end{split}
\end{equation}

Let $x_1=\mathbf{q}$, $x_2=\mathbf{\dot{q}}$, and $x=[x_1,x_2]^T$. Then we have the \textbf{standard form for linearization}, i.e. $\dot{x}=f(x,u)$
\begin{equation}
    \dot{x} = \left[ \begin{array}{c}
        x_2 \\ 
        \mathbf{H}(\mathbf{q})^{-1}(-\mathbf{C}(\mathbf{q},\mathbf{\dot{q}})\mathbf{\dot{q}} - \mathbf{G}(\mathbf{q}) + \mathbf{B}\mathbf{u}) \\ 
        \end{array} \right]
\end{equation}

Linearize around the equlibrium point $x^\ast=0$,
\begin{equation}
    \dot{x} = Ax + Bu
\end{equation}
where
\begin{align}
    A &= \left[ \begin{array}{cc}
        \mathbf{0} & \mathbf{I} \\ 
        -\mathbf{H}(\mathbf{q})^{-1}\frac{\partial}{\partial\mathbf{q}}\mathbf{G}(\mathbf{q}) & -\mathbf{H}(\mathbf{q})^{-1}\mathbf{C}(\mathbf{q},\mathbf{\dot{q}}) \\ 
        \end{array} \right], \\
        B &= \left[ \begin{array}{c}
            0 \\ 
            \mathbf{H}^{-1}\mathbf{B} 
            \end{array} \right]
\end{align}
Note that the term involving $\frac{\partial}{\partial\mathbf{q}}\mathbf{H}(\mathbf{q})^{-1}$ disappears because $\mathbf{C}(\mathbf{q},\mathbf{\dot{q}})\mathbf{\dot{q}} - \mathbf{G}(\mathbf{q}) + \mathbf{B}\mathbf{u}$ must be zero at the fixed point. Many of the $\mathbf{C}(\mathbf{q},\mathbf{\dot{q}})\mathbf{\dot{q}}$ derivatives drop out, too, because $\mathbf{\dot{q}}^\ast = 0$. The $-\mathbf{H}(\mathbf{q})^{-1}$ can be calculated by Matlab symbolic inversion,
\begin{equation}
    \mathbf{H}(\mathbf{q})^{-1} = \left[ \begin{array}{cc}
        \frac{1}{m_1+m_2-m_2\cos^2\theta} & \frac{-m_2\cos\theta}{m_1+m_2-m_2\cos^2\theta} \\ 
        \frac{-\cos\theta}{m_1l+m_2l-m_2l\cos^2\theta} & \frac{m_1+m_2}{m_1+m_2-m_2\cos^2\theta} \\ 
        \end{array} \right]
\end{equation}
and
\begin{equation}
    \frac{\partial}{\partial\mathbf{q}}\mathbf{G}(\mathbf{q}) = \left[ \begin{array}{cc}
        0 & 0 \\ 
        0 & -g\cos\theta \\ 
        \end{array} \right]
\end{equation}

Substitute  $x^\ast=0$ and rearrange $x$ to be $x=[x,\dot{x},\theta,\dot{\theta}]$, we have
\begin{equation}
    A = \left[ \begin{array}{cccc}
        0 & 1 & 0 & 0 \\ 
        0 & 0 & -\frac{m_2}{m_1}g & 0 \\ 
        0 & 0 & 0 & 1 \\
        0 & 0 & \frac{m_1+m_2}{m_1l}g & 0 
        \end{array} \right], B = \left[ \begin{array}{c}
            0 \\ 
            1/m \\
            0 \\
            -1/(m_1l)
            \end{array} \right]
\end{equation}

\bibliographystyle{IEEEtran}
\bibliography{ecl_bib}



\end{document}
